\pagestyle{empty}
\hspace{0pt}
\vfill

\textit{``The Lord God blessed her richly and crowned her with
the diadem of pure gold.''}
\medbreak
Almighty, Eternal God! We pray that following the noble
example of the Queen St. Hedwig, we may serve generously all our
brothers and sisters in their temporal and spiritual needs. Through
Christ our Lord. Amen.
\vfill
\textit{The Queen St. Hedwig by Augustyn Jodrzejczyk; Art Collection of
Jasna Góra, 1933}
\hspace{0pt}
\newpage
